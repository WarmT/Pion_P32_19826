\documentclass[11pt,a5paper,oneside]{book}

\usepackage[
	top=1cm,
	bottom=1cm,
	left=1.5cm,
	right=2.5cm,
	headheight=17pt,
	includehead,includefoot,
	heightrounded,
]{geometry}

\usepackage[finnish]{babel}
\usepackage[utf8]{inputenc}
\usepackage[T1]{fontenc}

\title{Pion. P 32:n sotapäiväkirja (19826)}
% http://digi.narc.fi/digi/slistaus.ka?ay=80360

\usepackage{longtable}
\usepackage{setspace}
\usepackage{romannum}
\usepackage{graphicx}
\usepackage{ulem}
\usepackage{siunitx}

\usepackage{hyperref}
\hypersetup{
    colorlinks=true,
    linkcolor=blue,
    filecolor=magenta,      
    urlcolor=blue,
}
\urlstyle{same}

\usepackage{xcolor}

\usepackage{changepage}
\usepackage{fancyhdr}
\usepackage{extramarks}
\pagestyle{fancy}
\fancypagestyle{headings}

%% L/C/R denote left/center/right header (or footer) elements
%% E/O denote even/odd pages

%% \leftmark, \rightmark are chapter/section headings generated by the 
%% book document class

%\fancyhead[C]{}
%\fancyhead[C]{\slshape Pion. P 32:n sotapäiväkirja}
\fancyhead[L]{}
\fancyhead[C]{Pion. P 32:n sotapäiväkirja 1941-1941 (19826)}
\fancyhead[R]{\firstxmark}
\fancyfoot[]{}
%\fancyhead[R]{\thepage}

\begin{document}

\extramarks{}{}

\begin{titlepage}
	\begin{center}
		\vspace*{1.5cm}
        \href{http://digi.narc.fi/digi/view.ka?kuid=3753911}{\Large Tulo 1264} \\
       	\vspace{1.5cm}
        \textbf{\Large Pion. P 32} \\
       	\vspace{1.5cm}
		\textbf{\Huge SOTAPÄIVÄKIRJA} \\
		\vspace{1.5cm}
		\Large 16.6.1941-17.10.1941 \\
   	\end{center}
\end{titlepage}

\extramarks{}{}
\section{Esipuhe}
Tämä sotapäiväkirja perustuu Kansallisarkiston digitaaliarkistosta löytyviin digitoituihin jaksoihin arkistoyksiköstä Pioneeripataljoona 32 1941-1941 (19826), joka löytyy osoitteesta: \\
\url{http://digi.narc.fi/digi/slistaus.ka?ay=80360} \\

\includegraphics[scale=0.45]{jaksot_19826.png}

Kunkin sivun ylätunnisteessa oikeall on kyseisen digitoidun jakson numero ja linkki siihen kansallisarkiston digitoituun jaksoon.

Mikäli sotapäiväkirjan digitoidusta jaksossa on ollut vaikeuksia sanojen/merkintöjen tulkinnassa niin ne kohdat on merkitty \textcolor{red}{punaisella}. Linkki alkuperäiseen digitoituun jaksoon auttaa selvittämään näiden merkintöjen oikean merkityksen.

Sitä mukaa kun tekstissä on tullut lyhenteitä tai sanoja, joita lukija ei välttämättä tiedä niin näitä on pyritty selventämään sivun alalaidassa olevilla alaviitteillä. Lisäksi jokainen paikannimi on mainittu alaviitteissä helpottaakseen lukijaa seuraamaan etenemistä kartalta. Alaviitteissä saattaa löytyä myös muuta asiaan liittyvää lisäinformaatiota. Näitä alaviitteitä ei ole alkuperäisessä sotapäiväkirjassa.
%\FloatBarrier

\newcommand{\taulustart}[2]{
\newpage\extramarks{\href{#1}{#2}}{}
\begin{center} % optimoitu A5 formaattiin:
\renewcommand*{\arraystretch}{1.4}
\begin{longtable}{ | p{0.14\textwidth} | p{0.09\textwidth} | p{0.77\textwidth} |}
\hline \multicolumn{1}{|p{0.14\textwidth}|}{\textbf{Päiväys}} & \multicolumn{1}{p{0.09\textwidth}|}{\textbf{Kello}} & \multicolumn{1}{p{0.77\textwidth}|}{\textbf{SISÄLTÖ}} \\ \hline
\endfirsthead
\hline \multicolumn{1}{|p{0.14\textwidth}|}{\textbf{Päiväys}} & \multicolumn{1}{p{0.09\textwidth}|}{\textbf{Kello}} & \multicolumn{1}{p{0.77\textwidth}|}{\textbf{SISÄLTÖ}} \\ \hline
\endhead \hline \endfoot \hline \endlastfoot}

\newcommand{\taulustop}{\hline \end{longtable} \end{center} \newpage}

\taulustart{http://digi.narc.fi/digi/view.ka?kuid=3753912}{2}

16.6.41 & 11.00 & Kokoontuminen Lohjan Kauppalan Yhteiskoululla\footnote{\href{https://www.google.fi/maps/place/Suurlohjankatu+2,+08100+Lohja/}{Suurlohjankatu 2, Lohja}}. Perustamispaikka ei ollut sopiva. Kun pataljoonan muodostavat reserviläiset olivat etupäässä lohjalaisia oli järjestyksen ylläpitäminen vaikeanlainen. Lisäksi vaikeutti perustamistapaikaksi valittu Yhteiskoulun pienuus järjestyksen ylläpitämistä. \\
% Lohjan Kauppalan Yhteiskoulu toimi aluksi vuokralla Anttilan talon vanhassa asuinrakennuksessa osoitteessa Suurlohjankatu 2, Lohja.

17.6.41 & 19.30 & Lähtivät kapt. Marrasmaa, luutn. Raimoranta, vänr. Helanto ja \textcolor{red}{ajomies} henkilöautolla Grabbskog träskiin\footnote{\href{https://www.google.fi/maps/place/Stortr\%C3\%A4sket/@60.0191777,23.3441542,15z}{Grabbskog Storträsket, Raasepori}} käskyn mukaisella tiedustelumatkalla yhteysottoon HR\footnote{Hangon Ryhmä}:ään.\\
% Träsk on ruotsia ja se on suomeksi järvi. Kartasta löytyy Grabbskog Storträsket Raaseporin läheisyydestä.

17.6.41 & 11.00 & Yksikköjen tuntolevyluettelot, ylimääräiset A- ja lääk. kortit lähetettiin SA-käskun mukaan Lohjan sk. toimistoon. Samoin luettelot puuttuvista korteista. 3 komppannian jako suoritettiin loppuun. Edelleen määrättiin lopullisesti kunkin yksikön päälliköt ja esikunnan kokoon- \\
\newpage

& & pano: Patalj. kom. Kapt. Marrasmaa,\newline adj. luutn. A. Gustafsson \newline pion. ups. luutn. I. Laatinen\newline talousups. luutn. S. Kärki \newline Autoups. luutn. J. Brax \newline 1.K.pääll. Ins. Kapt. Hotinen \newline 2.K.pääll. luutn. Korhonen \newline 3.K.pääll. luutn. Raimoranta \newline Valonheitinj. johtajaksi määrättiin vääpeli Huhtanen. \newline Raimorannan kompp. muodosti 13. Prikaatin pion. Komppaniat.\\

& 12.00 & Alkoi keskitysmarssi. Autorivistö lähti 12.07, johon kuului Esik., Valonheitinj. ja 2 K. johtajana luutn. Laatinen. \\

& 15.00 & Esik. ja 2 K-/ auto saapui Grabbskogträskiin. \\

& 16.15 & Lähetettiin 5 km auton hakemaan 1 K:aa \\

19.6.41 & 2.00 & Saapui 1 K:n miehet Leksvalliin\footnote{\href{https://www.google.fi/maps/place/10660+Leksvall/}{Leksvall, Raasepori}}. \\

& 7.30 & Hevoset saapuvat Grabbskogträskiin. \\

& 7.45 & Ensimmäinen valm. ilmoitus. \newline $a=\frac{137}{19}$, $b=\frac{84}{82}$, $e=\frac{92}{424}$, $d=\frac{92}{532}$ \\
\taulustop

\taulustart{http://digi.narc.fi/digi/view.ka?kuid=3753913}{3}

19.6. & 9.15 & 3 Komppanian au-jaettiin, 5 au 1 K:aan ja 3 2:aan. \\

19.6-20.6 & & Majoitusjärjestelyjä. Ei mitään erikoista. 1 H-auto, 1 \textcolor{red}{ku}-auto ja 1 mp lisää Lohjalta. Esikunnan joukot epäkunnossa. Korjattu iltaan mennessä. 1 K:n pääll. Kapteeni Hotinen siirrettiin HR:n \textcolor{red}{Tukikohtaan}. Samoin vänr. \textcolor{red}{Ilomin} ja Koivula. Edellämainittu Pion. toimistoon. \newline 1 Kompp. ollut miinoitustyössä \newline 2 Kompp. järjestynyt majoitukseen ja osa ollut tietyössä\\ 

21.6. & 7.00 & Vaun. maitoa \newline Kävin DE:ssä, Maj. Laakso antoi määräyksen ruuhikaluston\footnote{\url{https://elavamuisti.fi/aikajana/ruuhikaluston-kaytto}} kuntoon laittamisesta. En vieläkään saanut karttoja. Meidän on itse koulutettava erikoismiehistömme. Osa Pion. kalustosta siirrettiin Pion. Tp:hen. \underline{\textcolor{red}{Kanttiini}} valmistui. 11 miestä siirrettiin 3 K:sta 2 K:hon.\\

22.6. & 7.00 & valm. ilm. $a=137, b=87, e=102, d=100$ \\
\newpage
& 7.00 & Ilmoitukset \%:ssa kokovahvuudesta = 100\%, autovahvuus 31 \%, hevosvahvuus 100 \%.\\ 

& 16.10 & \underline{Pion. Kom. kirje 31}/Pion./\Romannum{2}/F/sal. Kompp. \newline Koulutuksesta: 11. ja 2 K noudatettava yleistä kertausharjoituskoulutusohjeita. 3 K:n koulutuksen painopiste tulee olla rynnäkköpion. koulutuksessa siten, että \textcolor{red}{u} joukkue pystyy väkivaltaiseen ylimenoon.\newline Pataljoonan on suoritettava tiedustelua mahdollista ylimenoa varten peitepiirroksessa mainitulla linjalla 1/A-B ja ryhdyttävä valmistamaan uivaa siltaa kohdassa L (Solböle\footnote{\href{https://www.google.fi/maps/place/10570+Solb\%C3\%B6le/}{Solböle, Raasepori}}) \newline Kirje 32/Tietetio/\Romannum{3}/sal. mukaan on patalj. ryhdyttävä tientekoon alkaen A:sta ja huomioimattaan sen, ettei muu pion. koulutus kärsi. Tie tehtävä yhteen suuntaan liikennöitäväksi autotieksi \textcolor{red}{sivuuten kohteisiin.} \\

23.6. & & Yhteys JR 55:en everstil. Wiberg. Keskusteltiin tien A-B rakentamisesta. Pyydettiin apua. Ei tule? \\

\taulustop

\taulustart{http://digi.narc.fi/digi/view.ka?kuid=3753914}{4}

23.6. & & Ilmoitettu komppanioille hälytysvalmiudesta. Vänr. \textcolor{red}{Sjöblom} 2 au pat. Haartmanin käyttöön. \\

& 18.00 & valm. ilm. $a=131, b=89, e=103, d=132$ \newline valm. ilm. 1) 102, 2) 49, 3) 100. \\

24.6. & & Valm. ilm. sama kuin ennen\newline 3 K. alistettu muualle.\newline Saatu uusia tehtäviä:\newline 1 K \underline{Miinakenttää} ei saa rakentaa lisää.\newline Radan korjausvälineet otettiin pois Skogbyhyn\footnote{\href{https://www.google.fi/maps/place/10680+Skogby/}{Skogby, Raasepori}} saakka. Skogby-järven ympäri tehtiin tie. Lehvallin laiturin viimeistely. Rautatien \textcolor{red}{poly} puolella oleva rataylimenopaikat valmistetaan.\newline \underline{Oikotie} tehtävä Österbyssä\footnote{\href{https://www.google.fi/maps/place/\%C3\%96sterby,+10620+Raasepori/}{Österby, Raasepori}}.\newline \underline{2 K.} Huoltotie PAp:n luota etelään.\newline Lautan siirtäminen Åminneforssin\footnote{\href{https://www.google.fi/maps/place/10410+\%C3\%85minnefors/}{Åminnefors, Raasepori}} padon alapuolelta yläpuolelle.\newline Solbölen ylikulkupaikka järjestettävä lautoilla. 2 lauttaa.\newline Saatu määräys maastl. ajoneuv. maalauksesta. \\

25.6. & & Palvelusohje no 1 saatu.\newline Valm. ilm. tavalliseen tapaan. Ei mitään erikoista. 2 K. siirtyi Troll-\\
\newpage
& & shovdaan\footnote{\href{https://www.google.fi/maps/place/10520+Trollshovda/}{Trollshovda, Raasepori}}. Lautan siirtäminen vaikeaa. Rakennetaan uusi lautta \textcolor{red}{kansineen työvälineillä}. \\

26.6. & & Patalj. sai kirjallisen tiedoituksen huoltotien loppuun rakentamisen siirymisestä JR 55:lle. Lautta valmis. \\

27.6. & & Saaru HR:n käsky no 1 kenttäröistä. Tehtiin henkilöstötäydennys: 1 \textcolor{red}{ueäk} au + 5 mekaanikkoa; 10 autonkulj. + 1 autonasentaja + 1 käsityöläinen.\newline Valm. ilm. tavalliseen tapaan. \\

28.6. & & Ei mitään erikoista. Kaiken aikaa jatkettu yleensä varustus- ja työvälineiden tasausta. Autojen täydennys ja huolto jatkuvasti heikkoa. Pataljoonan kuorma-autoista saatu ainoastaan 50\%. \\

29.6. & & Solbölen kumpikin lautta valmistui liikennekelpoisiksi. Koekuormitettu (8 tonnin kuorma) \\

30.6. & 6.30 & Pion. kom. määräyksestä 1 pion. ryhmä ups. \textcolor{red}{puolella} alistettiin Maj. Lagerlöfiin, \textcolor{red}{Armeijassa}.\newline Vänr. Murtomäki määrättiin johtajaksi. Tehtävänä raivata ryssien jättämä Horsön\footnote{\href{https://www.google.fi/maps/place/Hors\%C3\%B6n/@59.8909222,22.9598479,15z/}{Horsön, Raasepori}} saari\\

\taulustop

\taulustart{http://digi.narc.fi/digi/view.ka?kuid=3753915}{5}

& & miinoituksista ja ansoituksista. \\
& 9.20 & Ryhmä lähti.\newline 2 K. pääosa palasi Solbölestä.\newline Kirjallinen valm. ilmoitus.\newline \newline\newline\\

1.7.41 & & Solbölen asia selvä. Henkilösiirtoja toimitettu. 38 LK:aan siirretty 2 miestä; Kev.Os 19:ään 1+9 miestä.\\

& 20.30 & Murtomäen ryhmä palasi. Ei mitään erikoista.\newline Valm. ilm. tavalliseen tapaan.\newline 3 K., joka on alistettu muualle pataljoonan tuntemattomaan tehtävään, ei pysty tällä kertaa antamaan asianmukaista tilanneilmoituksia.\\

\newpage

2.7.41 & 10.00 & Ilmoittautui ev.ltn. 7. Ch. Fabritius. \newline Komentaja ja ev.ltn. F. seurasivat 2. K:n koulutusta aamupäiväläl. Iltapäivällä tilauksen esiesittely ev.ltn. Fabritiukselle. \\
& 19.05 & Osui yksi ryssän järeän tykistön ammus 40 m:n päähän Österbyn\footnote{\href{https://www.google.fi/maps/place/\%C3\%96sterby,+10620+Raasepori/}{Österby, Raasepori}} sk-talosta, josssa majaili 1/3 K. Aineelliset vahingot hyvin pienet.\newline Ei haavoittuneita. \newline\newline\newline\newline\\

3.7.41 & & Tasoituksia ja siirtoja 3 K. luovuttu 3 kuormallista kelvotonta materiaalia, rikkinisiä \textcolor{red}{ivt}. varusteita u. m. Pataljoona puolestaan luovuttanut ETp:hen ylimääräistä ja rikkinäistä materiaalia.\newline Koulutukset jatkettu.\\

\taulustop

\taulustart{http://digi.narc.fi/digi/view.ka?kuid=3753916}{6}

4.7. &  & Valm. ilmoitus tavalliseen tapaan. \\

& 20.30 & Saatu käsku 1. ja 2 K:n partioimisesta.\newline Väliraja maantie \textcolor{red}{Tsaari}-Hanko KK:hon asti. Oikealla 2. K. maantie pl., vasemmalla \underline{1.K} maantie. ml. Jatkuvasti tulee kaksi partiota kaistaa kohti olla toiminnassa.\\

& 18.00 & Käsky ev. ltn. Fabritiuksen komentamisesta os. Perksalon käyttöön yhdeksi viikoksi. \newline Pion. Kom. Laakson määräämät siirrot toimitettiin pienillä muutoksilla. Vänr. M. Kostamo määrättiin valonheitin joukk. johtajaksi. \\

\newpage

5.7 & & Ei mitään erikoista. Koulustusta jatkettu annettujen ohjeiden mukaan.\newline Valm. ilm. tavalliseen tapaan \newline\newline\newline\newline\newline\newline\newline \\ 

6.7 & 23.00 & Pataljoona lähetti 1/2 ryhmä seuraamaan ja avustamaan jv. partiota. Pion. 1/2 ryhmä oli 2. K:sta vänr. Lappalaisen johdolla (1 au + 2 pion.). Päätehtävänä oli tutustumienn maastoon sekä havaintojen teko vihollisen puoleisestamaastosta; laitteista sekä toiminnasta. Kapt Marrasmaa seurasi partiota etulinjalle saakka. Partion johtajan ja kapt. Marrasmaan selostukset liitetty sotapäiväkirjaan.\\

\taulustop

\taulustart{http://digi.narc.fi/digi/view.ka?kuid=3753917}{7}

& & Samana yönä osallistui myöskin 1.K:sta 4 miestä partioon. Luutn. J Haukinen selostus liitetty sotapäiväkirjaan.\newline\newline\newline\newline\\

7.7.41 & 23.00 & Yön toimintaan osallistui 2+10 pion, joista 2 pioneeria poartiossa radan suunnassa ja kaksi ryhmissä 1+4 jv:n mukana hyökkäyksessä 3 pion. haavoittui lievästi. nimittäin alik. \textcolor{red}{Eekuna}, pion. \textcolor{red}{Sahi} ja Kullberg. Lisäksi pion. Suominen paiskoitui \textcolor{red}{tuvan} paineesta viottaen kätensä. Luutn. J. Haukisen selostus liitetty sotapäiväkirjaan.\\ 
\newpage
 
 8.7 & & Ei mitään erikoista.\newline Yöllä taas partiointia. 1 K:sta 1+4 miestä. Kaikki sujui hyvin. Ei mitään mainittavaa.\newline\\ 

 & 22.30 & Määräys kapt. Hotiselta lähettää 14 kirvesmiestä laituri- ja lauttatyöhön Österbyssä\footnote{\href{https://www.google.fi/maps/place/\%C3\%96sterby,+10620+Raasepori/}{Österby, Raasepori}}. 1 K:sta ja 2 K:sta etettiin mummastakin 7 kirvesmiestä.\newline\\ 

 9.7 & & Österbyn työtä jatkettu. Yöllä partiointia. Ei mitään erikoista. Koulutusta jatketaan pääpainona iskuryhmien kouluttaminen.\\ 

\taulustop

\taulustart{http://digi.narc.fi/digi/view.ka?kuid=3753918}{8}

10.7. & 9.30 & Määräys lähettä yksi iskujoukkue Bromarfiin\footnote{\href{https://www.google.fi/maps/place/Bromarv/}{Bromarv, kunnan aiempi ruotsinkielinen nimi oli Bromarf.}}, Östanbergiin\footnote{\href{https://www.google.fi/maps/place/10570+\%C3\%96stanberg/}{Östanberg}}. 2 K:sta värn. Kahman johdolla lähetettiin klo 13.00 joukkue matkalle.\newline Yöllä tavallista partiointia.\newline Ei mitään erikoista.\newline Luutn Lindholm 2 K:sta määrättiin myös Bromarfiin yhdysupseeriksi.\\

11.7 & & \newline\newline\newline\newline\newline \\

11.7 & & Jatkuvaa partiointia. Ei mitään erikoista. Tuli käsku 30 miehen siirtämisestä Pion. Kolonnaan. Tilalle saatiin vastaava määrä.\newline Liekinheitinnäytös Skogbyn tunnelissa JR 13:ssa.\\ 
\newpage

& & \newline\newline\newline\newline\newline\newline \\

12.7 & 2.00 & Soitti Kahma ja ilmoitti, ettei pääse lähtemään Bromarfista.\newline Soitettuaan Laaksolle ja Kivelälle saatiin irroittamiskäsky.\newline Joukkue Kahma + liekinheitin \\

& 3.30 & ryhmä saapui \newline Luutn Lindholm + 1 joukkue 3 K:sta jäi senne.\newline\newline Tavanomaista partiointia. Ei mitään erikoista. Värn. Lappalainen (2 K:n partion joht.) kävi täällä ja ilmoitti jv:n \textcolor{red}{jättimessä} miehittämättä täältä irti etumaaston kohtia \textcolor{red}{m.m.} $X=4460, y=5690$ \\

\taulustop

\taulustart{http://digi.narc.fi/digi/view.ka?kuid=3753919}{9}

12.7.41 & 23.20 & Luutn. Lidholm saapui. 1 Joukkue 3 K:sta oli vihdoinkin saatu irrotettua. Luutn. L. ilmoitti joutuneensa joukkueen kanssa tulitaisteluun. 1 kaatunut, \newline Pion. Kaarlo Jokinen, Elimäki.\newline Ruumis toimitettu \textcolor{red}{Labbon} toimesta Kek:in. Luutn. L. selostus liitetty sotapäiväkirjaan. Niinikään vänr. Kahman selostus.\newline\newline\newline\newline\newline \\ 

13.7.41. & & Ei mitään erikoista. Vapaa päivä iskujoukkueiden miehille. Valm. ilmoitus tavalliseen tapaan.\newline \textcolor{red}{Illavuorolta} 1 K:ssa. tavalliseen tapaan partiointia. Ei mitään tuloksia. Ei pääse eteenpäin. \\
\newpage

& & \newline\newline\newline\newline \\

14.7.41 & 13.00 & Taisteluharjoitus 1K:n alueella.\newline Osaa otti 2 iskujoukkuetta. \newline Eversti Snellmann ja maj. Laakso sekä lukuisia muita upseereita oli seuraamassa.\newline Tavanomaista partiointia. Ei mitään erikoista. Koulutus jatkunut entiseen tapaan. \newline\newline\newline\newline\newline \\

15.7.41 & 13.00 & Taisteluharjoitus 2 K:n taisteluradalla. 1 iskujoukkue otti osaa. Harjoitus onnistui hyvin. Maj. Laakso oli tyytyväinen.\newline \\ 

& 20.30 & Kapt. M. käskettiin maj. L. puheille. \\

\taulustop

\taulustart{http://digi.narc.fi/digi/view.ka?kuid=3753920}{10}

& 21.45 & Puh. sanoma, että partiot heti vedettävä takaisin. \\

& 21.55 & Puh. sanoma yksiköille (1. ja 3 K.) k.o. asiasta. \\

& 23.45 & Saavui vänr. Lappalaisen partio (2 K.) 1 K:n partio ei ehtinyt lähteä matkalle. \newline\newline\newline\newline\newline \\

16.7. & \sout{16.7.} & Kiirettä: Kaikki välineet ja paperit saatava kuntoon. Henkilöstöasioista neuvoteltu Maj. L:n kanssa. \newline \\

& 18-19.00 & Rokotus\footnote{\textcolor{red}{Kurkkumätä, pilkkukuume vai joku muu?}} 2 K:ssa. Esikunta, val.h.j. ottivat myös osaa rokotukseen. \newline \\

& 20-21.00 & Rokotus 1 K:ssa. 3 K:aa ei roketeta koska ne on ennestään tänä aamuna rokotettu.\\
\newpage

& & Liik. tark. kortit lopullisesti lähetety pois ek piireille.\newline \\

& 22.30 & Käsky kuormauksesta. Vänr. \textcolor{red}{Aamin} toi sen. Kompp. pääll. kutsuttu tänne esikuntaan. \newline\newline\newline \\

17.7. & 11.30 & \sout{Alkoi} Lähti ensimmäinen kuorma-auto viemään tavaroita Pohjakurun asemalle\footnote{\href{https://www.google.fi/maps/place/60\%C2\%B005'53.9\%22N+23\%C2\%B033'07.7\%22E/}{Pohjakurun asema, Raasepori.}}. \\

& 17.05 & alkoi kuormaus. \\

& 20.00 & Kuormaus suoritettu. \\

& 20.20 & Juna lähti. Mukana Esikunta, valonh. joukkue + 1 K.\newline 2. ja 3 K seurasivat toisessa joka lähte klo 24.00. \newline \\

18.7. & & Matkapäivä. Ei mitään erikoista. \\	

\taulustop

\taulustart{http://digi.narc.fi/digi/view.ka?kuid=3753921}{11}

19.7. & 11.05 & Tulo Tohmajärven asemalle\footnote{\href{https://www.google.fi/maps/place/62\%C2\%B014'36.0\%22N+30\%C2\%B021'21.4\%22E/}{Tohmajärven asema}} \\

& 11.25 & Alkoi purkaus. \\

& 12.20 & Purkaus suoritettu \newline Esik. Valonh.j. ja 1 K. majoitusalueeksi määrättiin tienristeys Värtsilä-\textcolor{red}{Kutson}\footnote{\textcolor{red}{Tienristeystä ei löydy kartasta}} noin 5 km asemalta. \\

& 16.00 & Saapui 2 K ja 3 K. Majoittuivat samaan paikkain.\newline 1 mies 3 K:sta kadonnut matkalla \textcolor{red}{Stan} Tuominen, \textcolor{red}{Oj.} Viimeksi nähty Riihimäellä. \newline\newline \\

& 22.00 & Käytiin DE:ssä. Tavattiin ev. Snellmann ja ev.ltn. Sittkoff. \\
\newpage

20.7 & 10.00 & DE:n käsky muuttaa maj-paikka.\newline Majoituspiedusteluja. \newline \\

& 18.30 & Yksikön päälliköt komentajan puheille. Pataljoona siirtyy samana iltana Jänisjoen\footnote{\href{https://www.google.fi/maps/place/J\%C3\%A4nisjoki/@62.1921584,30.5864508,14z/}{Jänisjoki, Tohmajärvi}} rannalle majoittuen sen molemmilla puolella.\newline 3 K. itäpuolella, 1 ja 2 K. länsipuolella, Esikunta, val.h.j Saarion Kansakoululle\footnote{\href{https://www.google.fi/maps/place/62\%C2\%B015'10.2\%22N+30\%C2\%B028'56.7\%22E/}{Saarion kansakoulu on toiminut Saariovaaran rinteessä sijaitsevan Yläpihan päärakennuksessa}}. \newline \\

& 23.30 & Saapui ilmoitus, että komppaniat olivat majoittautuneet uusiin paikkoihin. \\

\taulustop

%\FloatBarrier
%\newpage

\end{document}